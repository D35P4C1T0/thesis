\section{Possibili ottimizzazioni lato Back-End}
\label{ch:flask-compress-test}

\subsection{Flask Compress}
La compressione dei dati trasmessi via rete è una tecnica fondamentale per migliorare le prestazioni delle applicazioni web, riducendo la latenza e il consumo di banda. Nel contesto di applicazioni sviluppate con il framework Flask, grazie all'utilizzo dell'estensione \texttt{Flask-Compress} si è in grado di abilitare la compressione (tipicamente \textit{gzip, deflate} o \textit{brotli}) delle risposte HTTP. Questo capitolo presenta un programma di test e uno script di supporto per valutare concretamente l'efficacia di \texttt{Flask-Compress} su diversi tipi di dati.
Per confrontare le prestazioni con e senza compressione e comprendere quanto beneficio possa portare questa estensione, è stata sviluppata una semplice applicazione Flask con due endpoint distinti: uno che serve dati senza alcuna compressione applicata da \texttt{Flask-Compress} e uno che serve gli stessi dati permettendo a \texttt{Flask-Compress} di intervenire.

Il codice Python per l'applicazione è il seguente:

\begin{longlisting}
\caption{Codice dell'applicazione Flask per il test di compressione}
\label{lst:flask_compression_app} % Ho aggiunto un'etichetta utile per riferimenti
\begin{minted}{python}
from flask import Flask, jsonify, make_response
from flask_compress import Compress
import random
import string
import json
import os

app = Flask(__name__)
# Inizializza Flask-Compress
# Di default, Flask-Compress comprimera' le risposte per text/html,
# text/css, text/xml, application/json, application/javascript, ecc.
# e solo per richieste che includono 'gzip' nell'header Accept-Encoding.
compress = Compress()
compress.init_app(app)

def generate_large_text(size_mb=1):
    """Genera una stringa grande di caratteri casuali."""
    chars = string.ascii_letters + string.digits + string.punctuation + ' '
    # Calcola il numero di caratteri necessario per la dimensione desiderata
    num_chars = size_mb * 1024 * 1024
    # Genera la stringa in blocchi per evitare un uso eccessivo di memoria
    text_chunks = [''.join(random.choice(chars) for _ in range(1024 * 10))] * (num_chars // (1024 * 10))
    remaining_chars = num_chars % (1024 * 10)
    if remaining_chars > 0:
        text_chunks.append(''.join(random.choice(chars) for _ in range(remaining_chars)))
    return ''.join(text_chunks)

def generate_compressible_json(num_entries=10000):
    """Genera un oggetto JSON con dati ripetitivi."""
    data = {
        "description": "Questo e' un oggetto JSON di test con dati ripetitivi per dimostrare l'efficacia della compressione.",
        "entries": []
    }
    for i in range(num_entries):
        data["entries"].append({
            "id": i,
            "name": f"Item {i % 100}", # Nomi ripetitivi
            "category": f"Category {(i % 5) + 1}", # Categorie ripetitive
            "value": random.uniform(0, 1000),
            "timestamp": "2000-01-24T04:40:00Z" # Timestamp ripetitivo
        })
    return data

def generate_less_compressible_data():
    """Genera dati meno comprimibili (es. byte casuali)."""
    # Nota: Flask-Compress si rivolge principalmente a tipi testuali.
    # Questo serve maggiormente a dimostrare che non comprimera' efficacemente dati non testuali.
    return os.urandom(1024 * 1024) # 1MB di byte casuali

# --- Route Flask ---

@app.route('/uncompressed')
def uncompressed_data():
    """Serve dati senza Flask-Compress."""
    print("Serving uncompressed data...")
    large_text = generate_large_text(size_mb=2) # 2MB di testo
    compressible_json = generate_compressible_json(num_entries=20000) # Piu' voci
    less_compressible = generate_less_compressible_data() # 1MB byte casuali (saranno codificati in base64 nel JSON)

    response_data = {
        "type": "uncompressed",
        "large_text": large_text,
        "compressible_json_part": compressible_json,
        "less_compressible_part": less_compressible.hex()
    }

    # Crea manualmente la risposta per assicurarsi che Flask-Compress non interferisca
    response = make_response(jsonify(response_data))
    response.headers['Content-Type'] = 'application/json'
    # Sono rimossi esplicitamente eventuali header di compressione per sicurezza
    if 'Content-Encoding' in response.headers:
        del response.headers['Content-Encoding']
    print("Uncompressed data served.")
    return response

@app.route('/compressed')
def compressed_data():
    """Serve gli stessi dati, Flask-Compress gestira' la compressione."""
    print("Serving compressed data...")
    large_text = generate_large_text(size_mb=2) # 2MB di testo
    compressible_json = generate_compressible_json(num_entries=20000) # Piu' voci
    less_compressible = generate_less_compressible_data() # 1MB byte casuali

    response_data = {
        "type": "compressed",
        "large_text": large_text,
        "compressible_json_part": compressible_json,
        # Includi dati meno comprimibili (saranno codificati in base64 nel JSON)
        "less_compressible_part": less_compressible.hex()
    }

    # Flask-Compress applichera' automaticamente la compressione se il client la supporta
    response = jsonify(response_data)
    print("Compressed data served.")
    return response
    
if __name__ == '__main__':
    app.run(debug=True, port=5000)
\end{minted}
\end{longlisting}


Questo script definisce due \textit{route}:
\begin{itemize}
    \item \texttt{/uncompressed}: Restituisce un oggetto JSON contenente testo, JSON strutturato e dati binari casuali. La risposta viene creata manualmente per evitare l'intervento di \texttt{Flask-Compress}.
    \item \texttt{/compressed}: Restituisce esattamente gli stessi dati. \texttt{Flask-Compress}, essendo inizializzato per l'applicazione, intercetterà questa risposta e la comprimerà se il client supporta la compressione (tramite l'header \texttt{Accept-Encoding}).
\end{itemize}
I dati generati includono una grande stringa di testo e un oggetto JSON con dati ripetitivi (altamente comprimibili), oltre a una sezione di byte casuali (meno comprimibili) inclusa come stringa esadecimale nel JSON per mostrare l'effetto su dati misti.

\subsection{Script Bash per il Confronto}
Per automatizzare il processo di chiamata agli endpoint e avere un'iniziale visualizzazione testuale del confronto delle dimensioni, è stato utilizzato uno script Bash. Questo script utilizza i comandi Linux \texttt{curl} per scaricare i contenuti dai due endpoint e \texttt{stat} per ottenere le dimensioni dei file risultanti. Ciò mette a disposizione un rapido strumento per comprendere l'efficacia di \texttt{Flask-Compress}.

Il codice dello script Bash è il seguente:

\begin{longlisting}
\caption{Script Bash per confrontare gli endpoint}
\label{lst:bash_compare_endpoints}
\begin{minted}{bash}
#!/bin/bash

# Indirizzo base della tua applicazione Flask
FLASK_APP_URL="http://127.0.0.1:5000"

# Nomi dei file di output temporanei
UNCOMPRESSED_FILE="uncompressed_output.json"
COMPRESSED_FILE="compressed_output.json"

echo "Test di confronto tra endpoint con e senza compressione..."
echo "-------------------------------------------------------"

# 1. Chiama l'endpoint non compresso e salva l'output
echo "Chiamata all'endpoint non compresso: ${FLASK_APP_URL}/uncompressed"
# Usiamo -s per la modalita' silenziosa (non mostra la barra di progresso)
# Usiamo -o per specificare il file di output
curl -s -o "$UNCOMPRESSED_FILE" "${FLASK_APP_URL}/uncompressed"

# Controlla se la chiamata e' andata a buon bene
if [ $? -eq 0 ]; then
    echo "Output non compresso salvato in $UNCOMPRESSED_FILE"
else
    echo "ERRORE: Impossibile recuperare l'output non compresso."
    exit 1
fi

echo "-------------------------------------------------------"

# 2. Chiama l'endpoint compresso (richiedendo gzip) e salva l'output
echo "Chiamata all'endpoint compresso: ${FLASK_APP_URL}/compressed (richiedendo gzip)"
# Usiamo -H per aggiungere l'header Accept-Encoding: gzip
curl -s -H "Accept-Encoding: gzip" -o "$COMPRESSED_FILE" "${FLASK_APP_URL}/compressed"

# Controlla se la chiamata e' andata a buon bene
if [ $? -eq 0 ]; then
    echo "Output compresso salvato in $COMPRESSED_FILE"
else
    echo "ERRORE: Impossibile recuperare l'output compresso."
    # Pulisci il file non compresso se esiste
    [ -f "$UNCOMPRESSED_FILE" ] && rm "$UNCOMPRESSED_FILE"
    exit 1
fi

echo "-------------------------------------------------------"

# 3. Confronta le dimensioni dei file
echo "Confronto delle dimensioni dei file:"

# Usiamo 'stat -c %s' per ottenere la dimensione in byte (Linux)
SIZE_UNCOMPRESSED=$(stat -c %s "$UNCOMPRESSED_FILE")
SIZE_COMPRESSED=$(stat -c %s "$COMPRESSED_FILE")

echo "Dimensione non compressa: ${SIZE_UNCOMPRESSED} byte"
echo "Dimensione compressa:      ${SIZE_COMPRESSED} byte"

# Calcola la percentuale di riduzione
if [ "$SIZE_UNCOMPRESSED" -gt 0 ]; then
    # Utilizza bc per calcoli in virgola mobile
    REDUCTION_BYTES=$((SIZE_UNCOMPRESSED - SIZE_COMPRESSED))
    REDUCTION_PERCENT=$(echo "scale=2; ($REDUCTION_BYTES * 100) / $SIZE_UNCOMPRESSED" | bc)
    echo "Riduzione dimensione:    ${REDUCTION_BYTES} byte (${REDUCTION_PERCENT}%)"
else
    echo "Impossibile calcolare la riduzione percentuale (dimensione non compressa e' 0)."
fi

exit 0
\end{minted}
\end{longlisting}

\subsection{Interpretazione dei Risultati}

Per valutare l'efficacia di \texttt{Flask-Compress} in modo grafico, si è raccolta una serie di rilevazioni utilizzando lo script Bash precedentemente descritto. Per ribadire il suo comportamento, registra per ciascun file scaricato:

\begin{itemize}
  \item la dimensione in byte del payload non compresso;
  \item la dimensione in byte del payload compresso;
  \item il calcolo della riduzione assoluta (in byte) e percentuale.
\end{itemize}

I risultati sono stati poi raccolti in un file CSV e sintetizzati nel grafico a barre raggruppate in \autoref{fig:compression_grouped}. 

In tale grafico si è voluto mettere a confronto, test dopo test, quanto spazio occupano i dati prima e dopo l'applicazione di \texttt{Flask-Compress}. Sull'asse orizzontale sono riportate in ordine cronologico e di dimensione tutte le prove effettuate: dalla più piccola \say{Testo 10 KB} fino al più corposo \say{Binario 10 MB}, passando per le varie voci JSON e i file binari. Le barre blu rappresentano la dimensione originale dei dati in byte, mentre le barre rosse mostrano la dimensione risultante dopo la compressione.  

Ad una prima analisi, per i file di testo e per i JSON la riduzione diventa via via più marcata al crescere del volume: le differenze tra barre blu e rosse si fanno più pronunciate man mano che passiamo dai \textit{kilobyte} ai \textit{megabyte}, segno che l'efficacia di \texttt{Flask-Compress} scala molto bene con \textit{dataset} più grandi. Al contrario, i file binari mostrano una percentuale di riduzione costante, indipendentemente dalla loro dimensione: questo suggerisce che, per dati già in formato compresso o con scarsa ridondanza, il guadagno rimane stabile. Tale disposizione dei dati in gruppi affiancati facilita l'analisi visiva immediata: è sufficiente un'occhiata per capire dove la compressione agisce in modo più intenso e dove, invece, il margine di miglioramento è limitato.

Andando più nello specifico, vi sono ulteriori tendenze:

\begin{enumerate}
  \item \textbf{Dati di testo plain} – Per i file contrassegnati come \say{Testo} (da 10 KB a 20 MB), la compressione mostra un guadagno progressivo: più il file è grande, più spazio viene risparmiato. Questo riflette la natura ridondante del testo, che \textit{Gzip} sa ottimizzare molto bene.
  
  \item \textbf{JSON strutturato} – Anche per i JSON con centinaia di voci la riduzione rimane elevata (tra l'87 \% e il 90 \%), grazie alla ripetitività dei nomi di campo e dei separatori. Si nota che, superate le 10 000 voci, la percentuale si stabilizza intorno al 90 \%, segno di un'efficacia consistente su strutture dati complesse ma regolari, come appunto i \textit{datapoints} mostrati sulla mappa.
  
  \item \textbf{File binari} – I file etichettati \say{Binario} evidenziano una riduzione costante di circa il 43 \% indipendentemente dalla dimensione. Questo indica che, su dati già compatti o con bassa ridondanza, \textit{Gzip} riesce a cavarsela decentemente ma non può emettere salti di compressione a livello di testo.
\end{enumerate}

Nel complesso, l'analisi dei dati raccolti conferma che \texttt{Flask-Compress} riduce in modo significativo il volume dei dati trasmessi dalla nostra applicazione, specialmente sui contenuti testuali e JSON. Questa riduzione si traduce in un minor utilizzo di banda e in tempi di caricamento più rapidi per l'utente finale, migliorando l'esperienza complessiva senza richiedere modifiche invasive al codice di generazione dei dati.

% grafici
\begin{figure}[!ht]
  \centering
  \begin{tikzpicture}
    \begin{axis}[
      width=1\textwidth,
      height=0.45\textwidth,
      ybar,
      bar width=4pt,
      enlarge x limits=0.03,
      ylabel={Dimensione (Byte)},
      xlabel={Tipologia dati},
      symbolic x coords={
        Testo 10KB,Testo 50KB,Testo 100KB,Testo 250KB,Testo 500KB,Testo 1MB,Testo 2MB,
        Testo 5MB,Testo 10MB,Testo 15MB,Testo 20MB,
        JSON 100 voci,JSON 500 voci,JSON 1000 voci,JSON 5000 voci,JSON 10000 voci,
        JSON 25000 voci,JSON 50000 voci,JSON 75000 voci,JSON 100000 voci,JSON 125000 voci,
        JSON 150000 voci,
        Binario 10KB,Binario 50KB,Binario 100KB,Binario 250KB,Binario 500KB,
        Binario 1MB,Binario 2MB,Binario 3MB,Binario 4MB,Binario 5MB,Binario 10MB
      },
      xtick=data,
      x tick label style={
        rotate=90,
        anchor=east,
        font=\tiny
      },
      grid=major,
      cycle list={
        {blue!70,fill=blue!30},
        {red!70,fill=red!30}
      },
      group style={
        group size=2 by 1,
        x descriptions at=edge bottom,
        horizontal sep=1pt
      },
      % Legend repositioned above the axis
      legend style={
        at={(0.5,1.15)},    % center under the plot, adjust the -0.25 as needed
        anchor=north,
        legend columns=2,
        /tikz/font=\footnotesize
      }
    ]
      % Serie non compressa
      \addplot+[
        nodes near coords,
        nodes near coords style={font=\tiny, rotate=90, anchor=west},
      ] table[
        col sep=comma,
        x=Descrizione Test,
        y={Dimensione Non Compressa (Byte)}
      ]{chapters/flask-compress/compression_results.csv};
      \addlegendentry{Non compresso}

      % Serie compressa
      \addplot+[
        nodes near coords,
        nodes near coords style={font=\tiny, rotate=90, anchor=west},
      ] table[
        col sep=comma,
        x=Descrizione Test,
        y={Dimensione Compressa (Byte)}
      ]{chapters/flask-compress/compression_results.csv};
      \addlegendentry{Compresso}
    \end{axis}
  \end{tikzpicture}
  \caption{Confronto tra dimensione non compressa e compressa per ciascun test.}
  \label{fig:compression_grouped}
\end{figure}
