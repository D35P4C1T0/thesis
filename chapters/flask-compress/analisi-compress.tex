\section{Possibili ottimizzazioni lato Back-End}
\label{ch:flask-compress-test}

\subsection{Flask-Compress}

La compressione dei dati trasmessi via rete è una tecnica essenziale per ottimizzare le prestazioni delle applicazioni web, riducendo latenza e consumo di banda. Nel contesto di Flask, l'estensione \texttt{Flask-Compress} consente di abilitare automaticamente la compressione delle risposte HTTP (tipicamente \textit{gzip}, \textit{deflate} o \textit{brotli}). 

Per valutarne l'efficacia, è stata sviluppata un'applicazione di test con due endpoint distinti: il primo invia i dati senza compressione, il secondo li serve sfruttando \texttt{Flask-Compress}. Tale confronto consente di misurare concretamente il miglioramento ottenuto in termini di riduzione del payload e tempi di trasferimento.


Link codice: \url{https://gitlab.com/matteogirardi/thesis-code-attachments/-/blob/main/TestCompressione.py?ref_type=heads}

% Il codice Python per l'applicazione è il seguente:

% \begin{longlisting}
% \caption{Codice dell'applicazione Flask per il test di compressione}
% \label{lst:flask_compression_app} % Ho aggiunto un'etichetta utile per riferimenti
% \begin{minted}{python}
% from flask import Flask, jsonify, make_response
% from flask_compress import Compress
% import random
% import string
% import json
% import os

% app = Flask(__name__)
% # Inizializza Flask-Compress
% # Di default, Flask-Compress comprimera' le risposte per text/html,
% # text/css, text/xml, application/json, application/javascript, ecc.
% # e solo per richieste che includono 'gzip' nell'header Accept-Encoding.
% compress = Compress()
% compress.init_app(app)

% def generate_large_text(size_mb=1):
%     """Genera una stringa grande di caratteri casuali."""
%     chars = string.ascii_letters + string.digits + string.punctuation + ' '
%     # Calcola il numero di caratteri necessario per la dimensione desiderata
%     num_chars = size_mb * 1024 * 1024
%     # Genera la stringa in blocchi per evitare un uso eccessivo di memoria
%     text_chunks = [''.join(random.choice(chars) for _ in range(1024 * 10))] * (num_chars // (1024 * 10))
%     remaining_chars = num_chars % (1024 * 10)
%     if remaining_chars > 0:
%         text_chunks.append(''.join(random.choice(chars) for _ in range(remaining_chars)))
%     return ''.join(text_chunks)

% def generate_compressible_json(num_entries=10000):
%     """Genera un oggetto JSON con dati ripetitivi."""
%     data = {
%         "description": "Questo e' un oggetto JSON di test con dati ripetitivi per dimostrare l'efficacia della compressione.",
%         "entries": []
%     }
%     for i in range(num_entries):
%         data["entries"].append({
%             "id": i,
%             "name": f"Item {i % 100}", # Nomi ripetitivi
%             "category": f"Category {(i % 5) + 1}", # Categorie ripetitive
%             "value": random.uniform(0, 1000),
%             "timestamp": "2000-01-24T04:40:00Z" # Timestamp ripetitivo
%         })
%     return data

% def generate_less_compressible_data():
%     """Genera dati meno comprimibili (es. byte casuali)."""
%     # Nota: Flask-Compress si rivolge principalmente a tipi testuali.
%     # Questo serve maggiormente a dimostrare che non comprimera' efficacemente dati non testuali.
%     return os.urandom(1024 * 1024) # 1MB di byte casuali

% # --- Route Flask ---

% @app.route('/uncompressed')
% def uncompressed_data():
%     """Serve dati senza Flask-Compress."""
%     print("Serving uncompressed data...")
%     large_text = generate_large_text(size_mb=2) # 2MB di testo
%     compressible_json = generate_compressible_json(num_entries=20000) # Piu' voci
%     less_compressible = generate_less_compressible_data() # 1MB byte casuali (saranno codificati in base64 nel JSON)

%     response_data = {
%         "type": "uncompressed",
%         "large_text": large_text,
%         "compressible_json_part": compressible_json,
%         "less_compressible_part": less_compressible.hex()
%     }

%     # Crea manualmente la risposta per assicurarsi che Flask-Compress non interferisca
%     response = make_response(jsonify(response_data))
%     response.headers['Content-Type'] = 'application/json'
%     # Sono rimossi esplicitamente eventuali header di compressione per sicurezza
%     if 'Content-Encoding' in response.headers:
%         del response.headers['Content-Encoding']
%     print("Uncompressed data served.")
%     return response

% @app.route('/compressed')
% def compressed_data():
%     """Serve gli stessi dati, Flask-Compress gestira' la compressione."""
%     print("Serving compressed data...")
%     large_text = generate_large_text(size_mb=2) # 2MB di testo
%     compressible_json = generate_compressible_json(num_entries=20000) # Piu' voci
%     less_compressible = generate_less_compressible_data() # 1MB byte casuali

%     response_data = {
%         "type": "compressed",
%         "large_text": large_text,
%         "compressible_json_part": compressible_json,
%         # Includi dati meno comprimibili (saranno codificati in base64 nel JSON)
%         "less_compressible_part": less_compressible.hex()
%     }

%     # Flask-Compress applichera' automaticamente la compressione se il client la supporta
%     response = jsonify(response_data)
%     print("Compressed data served.")
%     return response
    
% if __name__ == '__main__':
%     app.run(debug=True, port=5000)
% \end{minted}
% \end{longlisting}


Questo script definisce due \textit{route}:
\begin{itemize}
    \item \texttt{/uncompressed}: Restituisce un oggetto JSON contenente testo, JSON strutturato e dati binari casuali. La risposta viene creata manualmente per evitare l'intervento di \texttt{Flask-Compress}.
    \item \texttt{/compressed}: Restituisce esattamente gli stessi dati. \texttt{Flask-Compress}, essendo inizializzato per l'applicazione, intercetterà questa risposta e la comprimerà se il client supporta la compressione (tramite l'header \texttt{Accept-Encoding}).
\end{itemize}
I dati generati includono una grande stringa di testo e un oggetto JSON con dati ripetitivi (altamente comprimibili), oltre a una sezione di byte casuali (meno comprimibili) inclusa come stringa esadecimale nel JSON per mostrare l'effetto su dati misti.

\subsection{Script Bash per il Confronto}
Per automatizzare il processo di chiamata agli endpoint e avere un'iniziale visualizzazione testuale del confronto delle dimensioni, è stato utilizzato uno script Bash. Questo script utilizza i comandi Linux \texttt{curl} per scaricare i contenuti dai due endpoint e \texttt{stat} per ottenere le dimensioni dei file risultanti. Ciò mette a disposizione un rapido strumento per comprendere l'efficacia di \texttt{Flask-Compress}.

Link codice: \url{https://gitlab.com/matteogirardi/thesis-code-attachments/-/blob/main/BashConfrontoEndpoint.sh?ref_type=heads}

% Il codice dello script Bash è il seguente:

% \begin{longlisting}
% \caption{Script Bash per confrontare gli endpoint}
% \label{lst:bash_compare_endpoints}
% \begin{minted}{bash}
% #!/bin/bash

% # Indirizzo base della tua applicazione Flask
% FLASK_APP_URL="http://127.0.0.1:5000"

% # Nomi dei file di output temporanei
% UNCOMPRESSED_FILE="uncompressed_output.json"
% COMPRESSED_FILE="compressed_output.json"

% echo "Test di confronto tra endpoint con e senza compressione..."
% echo "-------------------------------------------------------"

% # 1. Chiama l'endpoint non compresso e salva l'output
% echo "Chiamata all'endpoint non compresso: ${FLASK_APP_URL}/uncompressed"
% # Usiamo -s per la modalita' silenziosa (non mostra la barra di progresso)
% # Usiamo -o per specificare il file di output
% curl -s -o "$UNCOMPRESSED_FILE" "${FLASK_APP_URL}/uncompressed"

% # Controlla se la chiamata e' andata a buon bene
% if [ $? -eq 0 ]; then
%     echo "Output non compresso salvato in $UNCOMPRESSED_FILE"
% else
%     echo "ERRORE: Impossibile recuperare l'output non compresso."
%     exit 1
% fi

% echo "-------------------------------------------------------"

% # 2. Chiama l'endpoint compresso (richiedendo gzip) e salva l'output
% echo "Chiamata all'endpoint compresso: ${FLASK_APP_URL}/compressed (richiedendo gzip)"
% # Usiamo -H per aggiungere l'header Accept-Encoding: gzip
% curl -s -H "Accept-Encoding: gzip" -o "$COMPRESSED_FILE" "${FLASK_APP_URL}/compressed"

% # Controlla se la chiamata e' andata a buon bene
% if [ $? -eq 0 ]; then
%     echo "Output compresso salvato in $COMPRESSED_FILE"
% else
%     echo "ERRORE: Impossibile recuperare l'output compresso."
%     # Pulisci il file non compresso se esiste
%     [ -f "$UNCOMPRESSED_FILE" ] && rm "$UNCOMPRESSED_FILE"
%     exit 1
% fi

% echo "-------------------------------------------------------"

% # 3. Confronta le dimensioni dei file
% echo "Confronto delle dimensioni dei file:"

% # Usiamo 'stat -c %s' per ottenere la dimensione in byte (Linux)
% SIZE_UNCOMPRESSED=$(stat -c %s "$UNCOMPRESSED_FILE")
% SIZE_COMPRESSED=$(stat -c %s "$COMPRESSED_FILE")

% echo "Dimensione non compressa: ${SIZE_UNCOMPRESSED} byte"
% echo "Dimensione compressa:      ${SIZE_COMPRESSED} byte"

% # Calcola la percentuale di riduzione
% if [ "$SIZE_UNCOMPRESSED" -gt 0 ]; then
%     # Utilizza bc per calcoli in virgola mobile
%     REDUCTION_BYTES=$((SIZE_UNCOMPRESSED - SIZE_COMPRESSED))
%     REDUCTION_PERCENT=$(echo "scale=2; ($REDUCTION_BYTES * 100) / $SIZE_UNCOMPRESSED" | bc)
%     echo "Riduzione dimensione:    ${REDUCTION_BYTES} byte (${REDUCTION_PERCENT}%)"
% else
%     echo "Impossibile calcolare la riduzione percentuale (dimensione non compressa e' 0)."
% fi

% exit 0
% \end{minted}
% \end{longlisting}

\subsection{Interpretazione dei Risultati}

Per valutare l'efficacia di \texttt{Flask-Compress}, sono state raccolte misurazioni tramite lo script Bash descritto, registrando per ciascun file:

\begin{itemize}
  \item dimensione in byte non compressa;
  \item dimensione in byte compressa;
  \item riduzione assoluta e percentuale.
\end{itemize}

I dati sono stati sintetizzati in un file CSV e rappresentati nel grafico a barre raggruppate (\autoref{fig:compression_grouped}). Le barre blu indicano la dimensione originale, quelle rosse la dimensione compressa. 

Dall'analisi emerge che la compressione migliora progressivamente con file testuali e JSON più voluminosi, mentre i file binari mostrano una riduzione costante, indipendente dalla dimensione. La rappresentazione a barre affiancate facilita la lettura immediata delle differenze.

Tendenze specifiche:

\begin{enumerate}
    \item \textbf{Dati di testo plain}: guadagno progressivo con l'aumentare della dimensione, grazie alla ridondanza intrinseca del testo.
    \item \textbf{JSON strutturato}: riduzione elevata (87–90\%) stabile oltre 10 000 voci, dovuta alla ripetitività dei campi e dei separatori.
    \item \textbf{File binari}: riduzione costante circa 43\%, indicativa della minore ridondanza dei dati.
\end{enumerate}

Complessivamente, \texttt{Flask-Compress} diminuisce significativamente il volume dei dati trasmessi, riducendo l'utilizzo di banda e i tempi di caricamento senza modifiche invasive al codice.

% grafici
\begin{figure}[!ht]
  \centering
  \begin{tikzpicture}
    \begin{axis}[
      width=1\textwidth,
      height=0.45\textwidth,
      ybar,
      bar width=4pt,
      enlarge x limits=0.03,
      ylabel={Dimensione (Byte)},
      xlabel={Tipologia dati},
      symbolic x coords={
        Testo 10KB,Testo 50KB,Testo 100KB,Testo 250KB,Testo 500KB,Testo 1MB,Testo 2MB,
        Testo 5MB,Testo 10MB,Testo 15MB,Testo 20MB,
        JSON 100 voci,JSON 500 voci,JSON 1000 voci,JSON 5000 voci,JSON 10000 voci,
        JSON 25000 voci,JSON 50000 voci,JSON 75000 voci,JSON 100000 voci,JSON 125000 voci,
        JSON 150000 voci,
        Binario 10KB,Binario 50KB,Binario 100KB,Binario 250KB,Binario 500KB,
        Binario 1MB,Binario 2MB,Binario 3MB,Binario 4MB,Binario 5MB,Binario 10MB
      },
      xtick=data,
      x tick label style={
        rotate=90,
        anchor=east,
        font=\tiny
      },
      grid=major,
      cycle list={
        {blue!70,fill=blue!30},
        {red!70,fill=red!30}
      },
      group style={
        group size=2 by 1,
        x descriptions at=edge bottom,
        horizontal sep=1pt
      },
      % Legend repositioned above the axis
      legend style={
        at={(0.5,1.15)},    % center under the plot, adjust the -0.25 as needed
        anchor=north,
        legend columns=2,
        /tikz/font=\footnotesize
      }
    ]
      % Serie non compressa
      \addplot+[
        nodes near coords,
        nodes near coords style={font=\tiny, rotate=90, anchor=west},
      ] table[
        col sep=comma,
        x=Descrizione Test,
        y={Dimensione Non Compressa (Byte)}
      ]{chapters/flask-compress/compression_results.csv};
      \addlegendentry{Non compresso}

      % Serie compressa
      \addplot+[
        nodes near coords,
        nodes near coords style={font=\tiny, rotate=90, anchor=west},
      ] table[
        col sep=comma,
        x=Descrizione Test,
        y={Dimensione Compressa (Byte)}
      ]{chapters/flask-compress/compression_results.csv};
      \addlegendentry{Compresso}
    \end{axis}
  \end{tikzpicture}
  \caption{Confronto tra dimensione non compressa e compressa per ciascun test.}
  \label{fig:compression_grouped}
\end{figure}
