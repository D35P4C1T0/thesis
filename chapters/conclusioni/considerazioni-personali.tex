\section{Considerazioni personali}
\label{ss:considerazioni-personali}

Lo sviluppo di questo progetto ha rappresentato per me un'occasione preziosa per approfondire tematiche legate alla visualizzazione di dati scientifici e all'ingegneria del software in ambito web. Al di là degli aspetti tecnici, ciò che ho trovato particolarmente stimolante è stato il dover progettare soluzioni non solo funzionali, ma anche usabili, robuste e sostenibili nel tempo.
\vspace{0.5em}

\noindent

Lavorare a un sistema pensato per essere importato, esteso e riutilizzato mi ha permesso di riflettere sulla necessità di scrivere codice chiaro, modulare e documentato. Ho imparato quanto sia importante definire buone astrazioni e mantenere una separazione netta tra logica di presentazione, configurazione e gestione dei dati. Inoltre, confrontarmi con strumenti avanzati di visualizzazione come \texttt{Deck.gl} mi ha aperto nuove prospettive su come i dati possano essere raccontati attraverso l'interazione visiva.

\vspace{0.5em}

\noindent

Uno degli aspetti che ho maggiormente apprezzato è stato il rapporto diretto con i dati geospaziali e ambientali. La possibilità di esplorarli in modo dinamico, adattando l'interfaccia alle esigenze di chi li consulta, mi ha fatto riflettere su come l'informatica possa facilitare la comprensione di fenomeni complessi anche al di fuori dell'ambito tecnico.

\vspace{0.5em}

\noindent
Naturalmente, non sono mancate le difficoltà: alcune scelte architetturali si sono rivelate più impegnative del previsto, e in certi casi ho dovuto modificare radicalmente l'approccio iniziale. Tuttavia, proprio in questi momenti ho avuto modo di migliorare la mia capacità di problem solving, di documentarmi in autonomia e di valutare criticamente le soluzioni disponibili.

\vspace{0.5em}

\noindent
In conclusione, considero questo lavoro non solo un'esperienza di sviluppo tecnico, ma anche un esercizio di progettazione consapevole, in cui teoria e pratica si sono intrecciate per dare vita a un sistema concreto, utile e potenzialmente riutilizzabile in altri contesti scientifici. Ne esco con una maggiore consapevolezza degli strumenti, ma anche con l'entusiasmo di continuare a imparare e contribuire a progetti complessi e multidisciplinari.