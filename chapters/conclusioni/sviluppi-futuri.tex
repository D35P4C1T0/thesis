\section{Sviluppi futuri}
\label{ss:sviluppi-futuri}

Il progetto costituisce un primo passo verso la creazione di un sistema integrato per la visualizzazione e l'analisi interattiva di dati geospaziali, con un focus particolare sul monitoraggio acustico marino. Dedico questa sezione all'analisi in maniera sistematica delle possibili evoluzioni del progetto, delineando sia gli sviluppi tecnici auspicabili nel breve e medio periodo, sia una roadmap di lungo termine in grado di guidare le future fasi di ricerca e sviluppo.

\subsection{Rinnovamento del Front-end}
Dal punto di vista dell'interfaccia, una migrazione a framework JavaScript moderni (come React o Vue.js) consentirebbe una maggiore modularità e manutenzione del codice. In prospettiva, il supporto a Progressive Web Apps (PWA) renderebbe l'applicazione utilizzabile anche in assenza di connessione.
L'ottimizzazione delle prestazioni grafiche passa per l'utilizzo avanzato di WebGL; in prospettiva, il supporto a WebGPU permetterebbe di sfruttare la potenza computazionale delle GPU direttamente nel browser. Aver optato per una mappa predisposta a tale tecnologia consentirà l'integrazione di WebGL senza ulteriori modifiche al codice originale.