\section{Analisi dell'Architettura}
\label{ch:backend-architecture}

\subsection{Factory Pattern e Dependency Injection}

Nel backend, è stato adottato il \emph{Factory Pattern} tramite la funzione \texttt{create\_heatmap\_blueprint()}, che genera dinamicamente \textit{blueprint} Flask configurati secondo parametri esterni. Ogni \textit{blueprint} funge da modulo indipendente, montabile o smontabile a seconda delle esigenze, garantendo coerenza e riusabilità del codice \cite{app-factories}. Il concetto di \textit{blueprint} è descritto in dettaglio nella sezione \ref{ss:blueprint}. In questo modo, il backend risulta modulare e adattabile a contesti applicativi diversi, riducendo la necessità di modifiche al codice ospitante.

In combinazione con il Factory Pattern, viene impiegata la \emph{dependency injection}, che permette di fornire componenti esterni (configurazioni, utility) al momento della creazione del modulo. Ciò evita l'uso di variabili globali, migliorando testabilità, flessibilità e manutenibilità del sistema \cite{dependency-injection-wiki,flask-di}.


\subsection{Struttura modulare - Blueprint}
\label{ss:blueprint}

Flask è un \textit{micro‑framework}, ma con l'aumentare della complessità dell'applicazione diventa cruciale adottare una struttura modulare per mantenere il codice chiaro e manutenibile.  
I \textit{blueprint} di Flask permettono di suddividere progetti complessi in componenti riutilizzabili, sviluppabili e mantenibili in modo indipendente, facilitando l'integrazione di nuovi moduli senza riscrivere il progetto principale.  
Essi consentono di raggruppare rotte, \textit{template} e file statici in \textit{mini}‑applicazioni all'interno della stessa istanza Flask. Non sono micro‑framework autonomi, ma \textit{insiemi di operazioni} da registrare sull'app principale, anche più volte.

I principali vantaggi dei \textit{blueprint} sono: \cite{palets_blueprints}

\begin{itemize}
  \item \emph{Modularità e incapsulamento}: ogni \textit{blueprint} racchiude logica e risorse relative a una specifica funzionalità (es. visualizzazione mappa, simulazione propagazione).
  \item \emph{Scalabilità}: consente di estendere l'app aggiungendo moduli senza influire sul codice esistente.
  \item \emph{Gestione centralizzata}: configurazioni e estensioni vengono condivise, pur mantenendo un'organizzazione chiara del progetto.
\end{itemize}


\subsubsection{Esempio di Blueprint}

Un \textit{blueprint} è un'entità modulare dichiarata come illustrato nel Listing \ref{lst:python_blueprint_registration} (es. \texttt{'simple\_page'} importato da \texttt{'templates'}), progettata per incapsulare rotte e risorse di un componente autonomo dell'applicazione.

Dopo la sua definizione, il \textit{blueprint} viene registrato nell'applicazione Flask principale tramite \texttt{app.register\_blueprint()} (Listing \ref{lst:python_blueprint_registration}). Questa operazione lo integra nel contesto runtime dell'applicazione.

Le rotte interne ai \textit{blueprints} sono gestite in modo relativo, prevenendo conflitti di naming con le rotte dell'applicazione principale o di altri \textit{blueprints}. Ciò significa che il percorso completo di una rotta interna dipende dal montaggio del \textit{blueprint}.

Questa gestione relativa permette di \say{montare} il \textit{blueprint} su un percorso \textit{radice} specifico (nell'esempio, su \texttt{/pages} tramite \texttt{url\_PREFIX}), creando un namespace URL logico. Una richiesta a \texttt{<indirizzo principale>/pages} viene quindi internamente instradata al \textit{blueprint}, consentendo un disaccoppiamento che favorisce la modularità e la riusabilità \cite{palets_blueprints}.

\begin{listing}
% \caption{Codice del file \texttt{simple\_page.py} che descrive un blueprint in Flask}
% \label{lst:flask-blueprint} % Il label va all'interno dell'ambiente listing o subito dopo la caption
\begin{minted}{python}
from flask import Blueprint, render_template, abort
from jinja2 import TemplateNotFound

simple_page = Blueprint('simple_page', __name__, template_folder='templates')

@simple_page.route('/<page>')
def show(page):
    try:
        return render_template(f'pages/{page}.html')
    except TemplateNotFound:
        abort(404)
\end{minted}
\caption{Codice del file \texttt{simple\_page.py} che descrive un blueprint in Flask (sopra) e Blueprint \texttt{simple\_page} montata in /pages}
\label{lst:python_blueprint_registration} % Aggiunto un nuovo label per questo secondo blocco
\begin{minted}{python}
from flask import Flask
from simple_page import simple_page

app = Flask(__name__)
app.register_blueprint(simple_page, url_PREFIX='/pages')
\end{minted}
\end{listing}

\subsection{Integrazione del blueprint}

Questo segmento illustra come, in un progetto Flask esterno, sia possibile integrare un sistema di heatmap completo con semplicità ed efficienza. L'approccio segue una filosofia \say{zero-configuration} per la configurazione di base, risultando adatto sia a prototipi rapidi sia ad applicazioni con requisiti standardizzati.

Il fulcro dell'integrazione è la funzione \texttt{register\_heatmap()}. Come mostrato nel codice seguente, basta invocarla passando l'istanza dell'applicazione Flask (\texttt{app}) per abilitare l'intera suite di funzionalità di heatmap. Questa astrazione della complessità consente agli sviluppatori di concentrarsi sulla logica principale del progetto, mentre la gestione del modulo heatmap viene completamente delegata.


\begin{listing}[H]
\caption{Esempio di Integrazione Semplificata della Heatmap in un Progetto Flask}
\label{lst:heatmap_simple_integration}
\begin{minted}{python}
from flask import Flask
from heatmap_blueprint import register_heatmap

app = Flask(__name__)

# Aggiungi qui le tue rotte esistenti
@app.route('/')
def home():
    return "My existing Flask app"

# Aggiungi la funzionalità heatmap
register_heatmap(app)

if __name__ == '__main__':
    app.run(debug=True)
\end{minted}
\end{listing}

A seguito di questa integrazione, supponendo di aver scelto un prefisso per il blueprint (rappresentato dal generico \texttt{URL\_PREFIX}), l'applicazione renderà disponibili automaticamente i seguenti endpoint HTTP, fornendo accesso all'interfaccia utente e ai servizi API della heatmap:

\begin{itemize}
    \item \texttt{http://localhost:5000/} -- Redirect alla dashboard principale.
\end{itemize}

\subsection{Endpoints con Prefisso} % Un sottotitolo per chiarezza
\label{ss:endpoints}

Tutti i seguenti endpoint sono preceduti da \texttt{http://localhost:5000/\{URL\_PREFIX\}/}:

\begin{itemize}
    \item \texttt{} -- Dashboard unificata con heatmap e visualizzazione della propagazione.
    \item \texttt{data} -- Restituisce i dati CSV predefiniti come JSON.
    \item \texttt{data/\{csv\_key\}} -- Restituisce uno specifico dataset come JSON.
    \item \texttt{csv-files} -- Restituisce l'elenco dei file CSV disponibili.
    \item \texttt{config-info} -- Restituisce le informazioni di configurazione e i percorsi auto-rilevati.
    \item \texttt{test-data} -- Testa il caricamento dei CSV e mostra la struttura dei file.
    \item \texttt{static/\{filename\}} -- Serve risorse statiche (auto-rilevate).
\end{itemize}

Questi esempi utilizzano \texttt{http://localhost:5000} per indicare l'indirizzo di accesso locale durante la fase di sviluppo dell'applicazione. L'hostname \texttt{localhost} si riferisce al server in esecuzione sulla macchina locale dello sviluppatore, mentre \texttt{5000} è la porta di default utilizzata dal server di sviluppo integrato di Flask. In un ambiente di produzione, questi URL sarebbero sostituiti dal dominio e dalla porta configurati per il deployment dell'applicazione (es. \texttt{https://mnat.org/heatmap/}).

\newpage


\subsection{Sistema di Configurazione}

Uno dei componenti centrali del sistema è la classe \texttt{HeatmapConfig}, che funge da interfaccia tra l'utente e le configurazioni interne del backend. Questo oggetto è in grado di accettare diverse forme di input: si può passare un singolo file CSV, una lista di file, oppure addirittura un dizionario con nomi personalizzati. Il merito di questa flessibilità è del metodo interno \texttt{\_process\_csv\_files()}, che normalizza i dati contenenti informazioni da disporre sulla mappa in una struttura facilmente gestibile dal resto dell'applicazione.

Link codice: \url{https://gitlab.com/matteogirardi/thesis-code-attachments/-/blob/main/HeatmapConfigClass.py?ref_type=heads}

% \begin{listing}[H]
% \caption{Costruttore della classe \texttt{HeatmapConfig}}
% \label{lst:heatmap_config}
% \begin{minted}{python}
% class HeatmapConfig:
%     def __init__(self, kwargs):
%         # Load global configuration defaults first
%         global_defaults = self._load_global_defaults()
        
%         # CSV file configuration - supports both single file and multiple files
%         self.INPUT_CSV_FILE = kwargs.get('INPUT_CSV_FILE', global_defaults.get('INPUT_CSV_FILE', 'data/data.csv'))
%         self.CSV_FILES = kwargs.get('CSV_FILES', None)  # New: dict of {name: filepath} or list of filepaths
%         self.DEFAULT_CSV = kwargs.get('DEFAULT_CSV', None)  # Which CSV to show by default
        
%         # Folder configuration
%         self.STATIC_FOLDER = kwargs.get('STATIC_FOLDER', global_defaults.get('STATIC_FOLDER', 'static'))
%         self.TEMPLATE_FOLDER = kwargs.get('TEMPLATE_FOLDER', global_defaults.get('TEMPLATE_FOLDER', 'templates'))
%         self.COLORS_DIR = kwargs.get('COLORS_DIR', 'colors')
        
%         # Data configuration
%         self.REQUIRED_COLUMNS = kwargs.get('REQUIRED_COLUMNS', global_defaults.get('REQUIRED_COLUMNS', ['Latitude', 'Longitude', 'Value']))
%         self.DEFAULT_MAP_OPACITY = kwargs.get('DEFAULT_MAP_OPACITY', global_defaults.get('DEFAULT_MAP_OPACITY', 0.75))
%         self.INITIAL_HEATMAP_RADIUS = kwargs.get('INITIAL_HEATMAP_RADIUS', global_defaults.get('INITIAL_HEATMAP_RADIUS', 40))
%         self.INITIAL_HEATMAP_INTENSITY = kwargs.get('INITIAL_HEATMAP_INTENSITY', global_defaults.get('INITIAL_HEATMAP_INTENSITY', 1.5))
%         self.INITIAL_HEATMAP_THRESHOLD = kwargs.get('INITIAL_HEATMAP_THRESHOLD', global_defaults.get('INITIAL_HEATMAP_THRESHOLD', 0.00))
        
%         # Blueprint configuration (these are blueprint-specific, no global defaults)
%         self.URL_PREFIX = kwargs.get('URL_PREFIX', '/heatmap')
%         self.BLUEPRINT_NAME = kwargs.get('BLUEPRINT_NAME', 'heatmap')
        
%         # Process CSV files configuration
%         self._process_csv_files()
% \end{minted}
% \end{listing}

È possibile scegliere come inizializzare la mappa: se con parametri di default, tramite file di configurazione o tramite parametri passati a \textit{runtime}. La priorità di ogni metodo è differente; abbiamo in ordine decrescente di priorità:

\begin{enumerate}
    \item Parametri passati a \texttt{register\_heatmap()}
    \item File di configurazione config.json globale
    \item Valori di default generici (failsafe)
\end{enumerate}
Nella pratica, questo significa che il sistema può essere utilizzato ed eseguito affidandosi a parametri di default senza una conoscenza profonda del progetto, sia tramite una configurazione granulare dei parametri iniziali. Personalmente, ho trovato molto comodo poter passare da una modalità all'altra senza dover modificare il codice, ma intervenendo unicamente sul file \texttt{config.json}.

\begin{listing}[H]
\caption{Metodi per inizializzare la mappa}
\label{lst:register_hm}
\begin{minted}{python}
# Istanza che usa config.json come base
register_heatmap(app, url_prefix='/global')

# Istanza che override specifici parametri
register_heatmap(
    app, 
    CSV_FILES={'Custom': 'data/custom.csv'},
    INITIAL_HEATMAP_RADIUS=60,      # ← Override global default
    url_prefix='/custom'
)
\end{minted}
\end{listing}

L'adozione di questa configurazione gerarchica comporta diversi benefici \cite{dependency-injection-wiki}:

\begin{itemize}
  \item \textbf{Modularità}: le dipendenze sono isolate dal codice, facilitando la sostituzione dei componenti.
  \item \textbf{Testabilità}: è possibile fornire istanze simulate (mock) in fase di test, evitando comportamenti indesiderati.
  \item \textbf{Scalabilità}: nuovi dataset o formati possono essere integrati senza cambiare la logica esistente.
\end{itemize}

L'architettura adottata garantisce un sistema robusto, estremamente flessibile e facilmente mantenibile, adatto sia a configurazioni base sia a scenari complessi e personalizzati.

\subsubsection{Gestione Multi-Dataset}
Il sistema prevede anche la gestione simultanea di più dataset CSV. Diversi formati di input (stringa singola, lista, dizionario) vengono normalizzati in una struttura uniforme.

\begin{listing}[H]
\caption{Dizionario multi-CSV} % Aggiunta una didascalia descrittiva
\label{lst:csv_normalization} % Un'altra etichetta univoca
\begin{minted}{python}
csv_files = {
    'Primary Dataset'  : 'data/data.csv',
    'Secondary Dataset': 'data/data2.csv',
    'Third Dataset'    : 'data/data3.csv'
}
\end{minted}
\end{listing}

A livello di configurazione, è possibile specificare più di un dataset fornendo un dizionario con la struttura in Listing \ref{lst:csv_normalization} come parametro di \texttt{register\_heatmap()}; un esempio di tale utilizzo è illustrato nel frammento di codice presente nel Listing \ref{lst:register_hm} (riga 6).
Fornire un dizionario simile durante l'inizializzazione di \texttt{HeatmapConfig} andrà ad influenzare il comportamento della mappa: sarà possibile selezionare quale Dataset visualizzare direttamente dai pannelli dell'interfaccia utente e il passaggio da uno all'altro avverrà dinamicamente, senza causare ricaricamenti della pagina.

\subsubsection{Configurazione tramite file \texttt{config.json}}

Il file \texttt{config.json} costituisce il centro di configurazione del sistema per la generazione della heatmap, permettendo di modificare rapidamente il comportamento dell'applicazione senza intervenire sul codice Python e rispettando il principio di separazione delle responsabilità.

Questo approccio modulare facilita la gestione dei parametri tra diversi ambienti (sviluppo, test, produzione), aumentando flessibilità, portabilità e leggibilità delle impostazioni \cite{understanding-config-json,config-files-types}.


Di seguito viene mostrata la struttura del file \texttt{config.json}:

\begin{listing}[H]
\caption{Contenuto del file \texttt{config.json}}
\label{lst:config_json} % Added a label for cross-referencing
\begin{minted}{json}
{
    "INPUT_CSV_FILE": "data/data.csv",
    "STATIC_FOLDER": "static",
    "FRONTEND_TEMPLATE": "index.html",
    "REQUIRED_COLUMNS": ["Latitude", "Longitude", "Value"],
    "DEFAULT_MAP_OPACITY": 0.75,
    "INITIAL_HEATMAP_RADIUS": 40,
    "INITIAL_HEATMAP_INTENSITY": 1.5,
    "INITIAL_HEATMAP_THRESHOLD": 0.00
}
\end{minted}
\end{listing}

Il file contiene sezioni ben definite che influiscono su aspetti fondamentali del sistema:

\paragraph{Configurazioni dei dati}  
\begin{itemize}
  \item \texttt{INPUT\_CSV\_FILE}: percorso del file CSV principale. Permette di cambiare dataset senza riconfigurare o ricompilare il sistema.
  \item \texttt{REQUIRED\_COLUMNS}: array con i nomi delle colonne obbligatorie. Viene eseguita una validazione automatica e consente di adattare velocemente l'ingestione dei dati a formati CSV differenti.
\end{itemize}

\paragraph{Configurazioni dell'interfaccia}  
\begin{itemize}
  \item \texttt{STATIC\_FOLDER}: specifica la cartella contenente risorse statiche (CSS, JS, immagini), rendendo più semplice l'integrazione con strutture front-end preesistenti.
  \item \texttt{FRONTEND\_TEMPLATE}: nome del file HTML principale, per definire facilmente interfacce personalizzate senza intervenire sul codice Python.
\end{itemize}

\paragraph{Configurazioni di visualizzazione}  
Controllano l'aspetto iniziale della heatmap:
\begin{itemize}
  \item \texttt{DEFAULT\_MAP\_OPACITY}: opacità della mappa base.
  \item \texttt{INITIAL\_HEATMAP\_RADIUS}, \texttt{INITIAL\_HEATMAP\_INTENSITY}, \texttt{INITIAL\_HEATMAP\_THRESHOLD}: definiscono rispettivamente i valori iniziali di raggio, intensità e soglia di visualizzazione, influenzando sia l'estetica che le prestazioni.
\end{itemize}



\subsection{Flusso Dati}

Durante lo studio delle svariate librerie di cartografia web, può emergere una falsa assunzione: i dati geospaziali in formato CSV, per essere importati all'interno di una libreria cartografica, debbano essere rigorosamente convertiti in \textbf{GeoJSON} per la visualizzazione, a prescindere dalla mappa scelta. Nel contesto di questo sistema, tale conversione \textbf{non avviene}; è stata esplicitamente evitata. La decisione non è frutto del caso, bensì di un ragionamento tecnico consapevole legato all'efficienza, alla semplicità e all'adattabilità dell'architettura.
Segue una tabella contenente informazioni riguardo la necessità di convertire i dati in formato GeoJSON per ogni libreria presa in esame:
\begin{table}[H]
\centering
\caption{Heatmap e supporto GeoJSON nelle librerie considerate}
\label{tab:geojson-heatmap-compact}
\begin{tabular}{@{}lll@{}}
\toprule
\textbf{Libreria} & \textbf{GeoJSON richiesto} & \textbf{Formato dati accettato} \\ \midrule
Leaflet & No & Array \texttt{[lat, lng, weight]} (via plugin) \\
Deck.gl & No & Oggetti JS con coordinate e peso \\
Folium & No & Lista o DataFrame (via Python) \\
Mapbox GL JS & Sì & GeoJSON con \texttt{type: \say{Feature}} \\
OpenLayers & Raccomandato & GeoJSON o oggetti \texttt{ol.Feature} \\
\bottomrule
\end{tabular}
\end{table}


Andando quindi per esclusione, risultano tre librerie in cui GeoJSON non è strettamente richiesto (Leaflet, Deck.gl, Folium). Tra tutte, Deck.gl e Folium supportano nativamente un formato di dati minimale, mentre Leaflet richiede software aggiuntivo. \cite{leaflet-doc, deckgl-docs, folium-doc, mapbox-docs, openlayers-doc}

\textbf{GeoJSON} è un formato aperto per lo scambio di dati geografici basato su JSON (JavaScript Object Notation) che definisce oggetti standardizzati per rappresentare entità spaziali e i loro attributi non spaziali. Si caratterizza per la semplicità sintattica e per l'adozione universale del sistema di riferimento WGS84, espresso in gradi decimali di longitudine e latitudine. La sua diffusione lo ha reso uno degli standard de facto per applicazioni web e GIS, grazie alla buona integrazione con librerie di \textit{mapping} e piattaforme di analisi spaziale \cite{geojson-spec,rfc7946}.

Il cuore operativo del backend è rappresentato dall'endpoint \texttt{/data/<csv\_key>}, che gestisce il caricamento, la validazione e l'esposizione dei dati al front-end. Quando si invoca questo endpoint, il sistema esegue una serie di operazioni concatenate:

\begin{itemize}
  \item verifica che il file esista e che l'utente abbia i permessi per accedervi,
  \item carica i dati tramite la funzione \texttt{pandas.read\_csv(..., index\_col=0)}, che consente di interpretare correttamente i file con indici espliciti, \cite{pandas-readcsv}
  \item calcola l'intervallo di valori (range), necessario per una corretta visualizzazione della \textit{heatmap},
  \item seleziona e applica la palette di colori per la \textit{heatmap}, con un fallback automatico se il file specificato non è disponibile.
\end{itemize}

\noindent Il risultato finale è un oggetto JSON già pronto per l'elaborazione da parte del front-end, senza necessità di ulteriori trasformazioni.

\newpage

\subsection{Benefici della Struttura Piatta}
\label{ss:why-not-geojson}

La trasformazione dei dati da DataFrame a JSON viene eseguita con una singola chiamata a \texttt{to\_dict()}, una funzione ottimizzata internamente dalla libreria \textit{Pandas}. Questo metodo sfrutta ottimizzazioni tali da evitare cicli espliciti \cite{pandas-performance}. La decisione di non utilizzare GeoJSON è stata dettata da un'attenta valutazione dei compromessi; è emersa la necessità di mantenere alta la performance, ridurre la complessità del codice e facilitare lo sviluppo e il debugging. L'uso di GeoJSON avrebbe richiesto di iterare il dataset riga per riga, creare manualmente le geometrie, sviluppare una serializzazione personalizzata e includere una validazione della struttura GeoJSON. Questo avrebbe complicato il codice e aumentato i tempi di risposta.

Il formato attuale è adatto non solo per heatmap, ma anche per visualizzazioni alternative come scatter plot, grafici a bolle o rappresentazioni tridimensionali. L'assenza di un formato rigido consente inoltre una rapida sperimentazione nel front-end.
Questo esempio mette in luce un aspetto importante dell'ingegneria del software: saper adattare le scelte tecniche al contesto, anche quando ciò significa discostarsi da soluzioni generalmente consigliate. Non si tratta di ignorare uno standard consolidato, ma di riconoscerne i limiti e valutare in modo critico la sua applicabilità. L'adozione di un formato piatto e compatto, invece del più articolato GeoJSON, ha portato a un'architettura più leggera, efficiente e perfettamente aderente agli obiettivi progettuali.

Nel corso dello sviluppo e dei test, il formato piatto si è rivelato estremamente utile: è immediatamente leggibile, sia nella console del browser che durante il \textit{logging} lato server. Questo ha permesso di ridurre il tempo necessario all'identificazione di problemi nei dati o nella mia implementazione.


\subsection{Flusso dei Dati nel Backend}

Riprendendo le rotte presenti nella sezione \ref{ss:endpoints}, all'interno dell'endpoint \texttt{/data/<csv\_key>}, i dati vengono letti da file CSV tramite la libreria \texttt{pandas}. Il codice seguente riporta il comportamento effettivo del backend:

\begin{listing}[H]
\caption{Lettura e serializzazione dei dati CSV}
\label{lst:csv_reading_serialization} % Ho aggiunto un'etichetta per riferimento
\begin{minted}{python}
df = pd.read_csv(csv_file, index_col=0)
data = df[config.REQUIRED_COLUMNS].to_dict(orient='records')

response_data = {
    'data': data,
    'valueRange': value_range,
    'colorScale': colors,
    'csvKey': csv_key,
    'csvFile': csv_file
}
return jsonify(response_data)
\end{minted}
\end{listing}

I dati vengono poi convertiti in una lista di dizionari Python utilizzando il metodo\\ \texttt{to\_dict(orient='records')} e serializzati in formato JSON. Tale passaggio risulta efficiente per la serializzazione \cite{pandas-todict}. La struttura risultante è piatta e leggibile; segue un esempio di un singolo datapoint in formato JSON:

\begin{listing}[H]
\caption{Formato semplificato JSON}
\label{lst:simplified_json_format} % Added a label for reference
\begin{minted}{json}
{
  "Latitude": 37.2589,
  "Longitude": -59.3,
  "Value": 88.046
}
\end{minted}
\end{listing}

\subsubsection{GeoJSON: Quando e Perché Non Usarlo}

Il formato GeoJSON è largamente utilizzato per rappresentare entità geografiche complesse come poligoni, linee o collezioni di feature. 
Tuttavia, nel caso specifico di dataset composti esclusivamente da punti con attributi numerici, il formato GeoJSON introduce un sovraccarico strutturale (\textit{overhead}) non necessario:

\begin{listing}[H]
\caption{Struttura GeoJSON equivalente}
\label{lst:geojson_equivalent_structure} % Added a label for reference
\begin{minted}{json}
{
  "type": "Feature",
  "geometry": {
    "type": "Point",
    "coordinates": [-59.3, 37.2589]
  },
  "properties": {
    "Value": 88.046
  }
}
\end{minted}
\end{listing}

La struttura semplificata usata nel progetto si è dimostrata, in termini di dimensione del JSON e della sua versione GeoJSON, fino al 70\% più compatta. Questo è particolarmente importante quando si lavora con decine di migliaia di punti da trasmettere al client. Nella sezione \ref{ss:why-not-geojson} si parla in maniera più dettagliata dei motivi dietro tale scelta implementativa.

GeoJSON rimane comunque un formato valido in molti contesti. La sua adozione sarebbe raccomandabile in scenari in cui è necessario rappresentare:

\begin{itemize}
  \item Poligoni e aree di interesse geospaziale;
  \item Linee di migrazione, rotte o percorsi;
  \item Confini amministrativi;
  \item Collezioni eterogenee di oggetti geografici.
\end{itemize}

\noindent Nel caso in esame, tuttavia, i dati consistono esclusivamente in punti con attributi associati, per cui un formato più leggero è preferibile.

\newpage

\section{Modularità e Riusabilità}

L'idea di fondo dell'intero progetto è quella di creare un sistema riusabile e facilmente integrabile in altri progetti. A questo scopo, la funzione \texttt{register\_heatmap()}, già apparsa nel frammento di codice nel Listing \ref{lst:register_hm}, offre un'interfaccia semplificata che permette di integrare il progetto in una qualsiasi applicazione Flask già esistente, senza dover modificare nulla a livello di struttura.

Ogni \textit{blueprint} mantiene il proprio namespace isolato, evitando conflitti tra percorsi API e risorse statiche. Questa separazione si è dimostrata particolarmente utile nella fase di integrazione di questo modulo con un'applicazione già esistente, poiché non è stato necessario modificare le rotte precedenti.

\subsection{Architettura e Funzionamento}

La versatilità di \texttt{register\_heatmap()} si manifesta attraverso diversi livelli di integrazione. All'interno di \texttt{register\_heatmap()} viene chiamata la funzione \texttt{create\_heatmap\_blueprint()}; tale funzione accetta una configurazione (\texttt{HeatmapConfig} o \texttt{dict}) e restituisce un \textit{blueprint} Flask configurato per la visualizzazione di heatmap.

Al suo interno viene creato un oggetto \texttt{Blueprint}:

\begin{listing}[H]
\label{lst:create_heatmap_blueprint}
\begin{minted}{python}
    # Create blueprint
    bp = Blueprint(
        config.BLUEPRINT_NAME,
        __name__,
        url_prefix=config.URL_PREFIX,
        template_folder=config.TEMPLATE_FOLDER,
        static_folder=config.STATIC_FOLDER,
        static_url_path=f'{config.URL_PREFIX}/static'
    )
\end{minted}

Nello stesso file vengono dichiarate le rotte del blueprint. Seguono degli esempi di rotte dichiarate tramite decoratore \texttt{'@'}. Nella rotta \say{\texttt{/data}} si fa uso di diverse funzionalità offerte da Flask: per una stessa funzione \texttt{get\_data()} vengono definite due rotte (\texttt{/data} e \texttt{/data/<csv\_key>}), semplificando il codice e rendendolo più leggibile. In aggiunta, il parametro opzionale \texttt{<csv\_key>} viene estratto dall'URI in maniera elegante e minimale (riga 6) e viene passato al corpo della funzione \texttt{get\_data()}.

\begin{minted}{python}
@bp.route('/')              # Pagina principale che carica la mappa
def index():
    return render_template('map.html', config=config)
    
@bp.route('/data')          # Endpoint per fornire i dati
@bp.route('/data/<csv_key>')
def get_data(csv_key=None):
    # Carica il CSV richiesto (o uno predefinito)
    # Estrae e converte i dati in JSON
    # Calcola il range dei valori e seleziona la palette colori
    # Restituisce i dati pronti per la heatmap
\end{minted}
\end{listing}

Il termine URI sta per \textit{Uniform Resource Identifier}. È una stringa di caratteri che identifica in modo univoco una risorsa su una rete (in questo caso, il web).
Quando parliamo di URI in Flask, ci riferiamo principalmente al percorso (path) della risorsa all'interno dell'URL. 
Ad esempio, in \texttt{https://website.com/blog/post/123}, il percorso \texttt{/blog/post/123} è l'URI che identifica quella specifica risorsa (il post numero 123 del blog).
La sua convenienza risiede nella capacità di gestire più varianti di un percorso con una singola funzione (handler), permettendo di ottenere dati sia generali che specifici con la stessa logica di base.

Il fulcro di questa semplicità è la funzione \texttt{register\_heatmap()}. Con una singola linea di codice, come mostrato nel Listing \ref{lst:zero_config_integration}, l'intera suite di funzionalità Heatmap viene iniettata nell'applicazione Flask. Questo approccio riduce drasticamente lo sforzo di setup, eliminando la necessità di configurare manualmente rotte, endpoint API o la gestione dei file di dati. Lo sviluppatore può così concentrarsi sulla logica principale dell'applicazione, delegando la complessità della Heatmap a un'unica chiamata.

\begin{listing}
\caption{Integrazione immediata con \texttt{register\_heatmap()}}
\label{lst:zero_config_integration}
\begin{minted}{python}
from flask import Flask
from heatmap_blueprint import register_heatmap

app = Flask(__name__)

# Una sola linea per integrare tutto il sistema
register_heatmap(app, csv_file='data/my_data.csv')

# Ora la tua app ha automaticamente:
# - /heatmap/ → Interfaccia web completa
# - /heatmap/data → API dati JSON
# - /heatmap/propagation → Simulazione onde sonore
# - /heatmap/csv-files → Gestione multi-dataset
\end{minted}
\end{listing}


\subsection{Flessibilità di Configurazione}
Mentre la funzionalità \say{Zero Configuration} di \texttt{register\_heatmap()} consente un'integrazione immediata, il vero punto di forza di questo sistema risiede nella sua flessibilità di configurazione. La funzione \texttt{register\_heatmap()} non si limita a fornire un'integrazione base, ma offre una personalizzazione completa attraverso un set esaustivo di parametri di configurazione. 
Questo permette agli sviluppatori di esercitare un controllo granulare su ogni aspetto della visualizzazione della Heatmap e della sua interazione con i dati. 
Tutto ciò viene raggiunto mantenendo la semplicità intrinseca di Flask. I parametri di configurazione sono passati direttamente come argomenti alla funzione \texttt{register\_heatmap()}, offrendo un'alternativa a file di configurazione esterni o al ricorrere a \textit{dependency injection}.

Link codice: \url{https://gitlab.com/matteogirardi/thesis-code-attachments/-/blob/main/HeatmapParameters.py?ref_type=heads}

% \begin{listing}[H]
% \caption{Parametri di personalizzazione completa per la heatmap}
% \label{lst:full_customization_params}
% \begin{minted}{python}
% register_heatmap(
%     app,
%     # === DATA CONFIGURATION ===
%     CSV_FILES={
%         'Survey Alpha': 'data/alpha.csv',
%         'Survey Beta': 'data/beta.csv'
%     },
%     DEFAULT_CSV='Survey Alpha',
    
%     # === VISUAL CONFIGURATION ===
%     HEATMAP_RADIUS=25,          # Raggio dei punti heatmap
%     HEATMAP_INTENSITY=1.5,      # Intensità visiva
%     HEATMAP_THRESHOLD=0.05,     # Soglia di visualizzazione
%     HEATMAP_OPACITY=0.8,        # Opacità overlay
%     COLOR_SCALE='viridis',      # Schema colori
    
%     # === MAP CONFIGURATION ===
%     MAP_CENTER_LAT=42.3601,     # Centro mappa iniziale
%     MAP_CENTER_LON=-71.0589,
%     MAP_ZOOM=8,                 # Zoom iniziale
    
%     # === UI CONFIGURATION ===
%     TITLE='My Custom Heatmap',  # Titolo interfaccia
    
%     # === BLUEPRINT CONFIGURATION ===
%     url_prefix='/my-heatmap',   # URL base
%     blueprint_name='my_heatmap' # Nome interno blueprint
% )
% \end{minted}
% \end{listing}


\subsection{Considerazioni sull'architettura}

L'architettura implementata non è certo priva di limiti, ma si è dimostrata efficace per gli obiettivi di questo lavoro. Il sistema è sufficientemente flessibile da poter essere adattato a scenari diversi, ed è organizzato in modo modulare per facilitare l'estensione futura.


In particolare, l'uso del Factory Pattern e della dependency injection ha reso lo sviluppo più ordinato, mentre la gestione centralizzata delle configurazioni e dei dataset ha permesso di operare con file eterogenei senza richiedere modifiche sostanziali. Tutti questi elementi, sebbene piuttosto semplici presi singolarmente, hanno contribuito a costruire un backend che potrà essere integrato in ambienti di lavoro più complessi.
