\section{Frontend: Visualizzazione Della Mappa}
\label{ch:frontend-advanced}

\subsection{Libreria scelta - Deck.gl}

Per il progetto si è optato per \texttt{Deck.gl}, una libreria avanzata basata su WebGL per la visualizzazione di dati geospaziali. Deck.gl consente di operare direttamente su array di oggetti JavaScript (\texttt{data}) senza necessità di conversione in GeoJSON \cite{deckgl-heatmap}.

Questa gestione semplificata dei dati mantiene l'interazione rapida e reattiva anche con dataset voluminosi, sfruttando la GPU per il rendering e garantendo fluidità con mappe complesse. Tale performance è fondamentale per l'esplorazione di dataset ad alta densità, come quelli di rumore subacqueo \cite{deckgl-heatmap}.

Deck.gl offre inoltre:
\begin{itemize}
  \item \textbf{Layer specializzati}, inclusi \texttt{HeatmapLayer}, per visualizzazioni tematiche \cite{deckgl-docs};
  \item \textbf{Supporto futuro a WebGPU}, con potenziali prestazioni superiori \cite{deckgl-roadmap};
  \item \textbf{Strumenti di monitoraggio delle prestazioni}, utili per sviluppo e ottimizzazione.
\end{itemize}


\subsection{Tema della UI}

Per ottenere un'interfaccia visivamente leggera e moderna è stato adottato lo stile \textbf{glassmorphism}, che riproduce l'effetto di superfici di vetro smerigliato sovrapposte \cite{glassmorphism-css}. La palette cromatica sobria, unita alla riduzione dei dettagli superflui, privilegia la leggibilità del testo e la chiarezza visiva. Colori, ombre, sfocature e contrasti vengono utilizzati come strumenti funzionali di comunicazione visiva.

Dal punto di vista tecnico, il design si basa su filtri CSS moderni (\texttt{backdrop-filter}), trasparenze e ombre morbide, ottenendo profondità e stratificazione senza incidere significativamente sulle prestazioni. Tale efficienza è cruciale, poiché le risorse principali devono essere riservate al rendering della mappa e dei dati. Esempi tratti dall'implementazione sono riportati in Figura~\ref{fig:glass-overview}.


\begin{figure}[H]
    \centering
    % Adjust height here to make the figure shorter
    \includegraphics[height=0.3\textheight]{images/UI-glass-1.png}%
    \hspace{1em}%
    \includegraphics[height=0.3\textheight]{images/UI-glass-2.png}
    \caption{\small Esempi di interfaccia con stile Glassmorphism.}
    \label{fig:glass-overview}
\end{figure}

\subsection{Design Responsivo}

Durante lo sviluppo si è posto l'accento sulla capacità del sito di adattarsi efficacemente al contenitore di visualizzazione. La presenza di una mappa interattiva e di pannelli con input eterogenei ha reso essenziale definire regole di \textit{layout} chiare e coerenti.

Il partizionamento ordinato dello spazio ha garantito un'interfaccia leggibile e fruibile, indipendentemente dalle dimensioni dello schermo o dal livello di zoom, assicurando un'interazione fluida in tutte le condizioni.

Per questo motivo si è scelto di basare il layout su \textit{Tailwind CSS}, integrato con \textit{media queries} personalizzate, che permettono di:
\begin{itemize}
  \item Ridimensionare automaticamente griglie e container in base alla larghezza del viewport;
  \item Adattare la disposizione dei controlli per una fruizione ottimale su dispositivi mobili \cite{tailwind-responsive};
  \item Mantenere consistenza stilistica tramite le \textit{utility class} predefinite \cite{tailwind-docs}.
\end{itemize}

Questo approccio garantisce un'esperienza utente uniforme su qualsiasi dispositivo.

\begin{figure}
    \centering
    \includegraphics[width=0.6\linewidth]{images/whole-map-homepage.png}
    \caption{Interfaccia Web della Heatmap}
    \label{fig:heatmap-homepage}
\end{figure}


\subsection{Controlli Dinamici}

Per un'esplorazione interattiva dei dati sono stati implementati \textit{widget} con pulsanti e slider che modificano in tempo reale la visualizzazione (Figura \ref{fig:glass-overview}). Questi controlli permettono di regolare:

\begin{itemize}
  \item \textbf{Intensità Heatmap}: opacità della heatmap per visualizzare la mappa sottostante;
  \item \textbf{Raggio di Influenza}: estensione dei pixel influenzati da ogni punto, modulando la densità visiva;
  \item \textbf{Soglia di Visualizzazione}: filtro dei punti sotto una soglia, riducendo il \say{rumore};
  \item \textbf{Selezione Dataset}: passaggio istantaneo tra serie di dati senza ricaricare la pagina (Figura \ref{fig:glass-overview}).
\end{itemize}

I controlli sono collegati ai layer di Deck.gl, garantendo aggiornamenti immediati, e possono essere nascosti per una migliore visualizzazione della mappa. Il sistema integra funzionalità pensate per ottimizzare l'interazione e l'analisi dei dati, offrendo sia una panoramica generale che analisi dettagliate con un'interfaccia intuitiva.

\subsubsection*{Value Picker} 
Attivabile tramite \say{Enable picker mode}, consente di selezionare punti specifici sulla mappa. Passando con il cursore su un punto appare un riquadro con coordinate geografiche (Latitudine/Longitudine) e valore numerico preciso, copiabile tramite click. Questo supporta l'analisi puntuale dei dati.

\subsubsection*{Statistiche Dati}
Fornisce una panoramica sintetica del dataset: \say{Total Points} indica il numero totale di datapoint, mentre \say{Data Range} mostra intervallo minimo e massimo, utile per comprendere la scala dei fenomeni.

\begin{figure}[H]
    \centering
    \includegraphics[width=0.3\linewidth]{images/legenda.png}
    \caption{Legenda}
    \label{fig:legenda}
\end{figure}

\subsubsection*{Legenda Colori}
Facilita l'interpretazione della heatmap tramite un \say{Gradiente dinamico} che si adatta al \say{Range valori} del dataset (Figura \ref{fig:legenda}). Un pulsante consente di mostrare o nascondere la legenda, mentre le etichette numeriche indicano i valori minimo e massimo, essenziali per l'interpretazione quantitativa.


\subsection{Simulazione Propagazione}

Il sistema adotta un modello di propagazione acustica fondato sui principi fisici dell'acustica subacquea. La velocità di propagazione è fissata a 1481 m/s, un valore di riferimento comunemente accettato per la velocità del suono in acqua marina, corrispondente a condizioni standard di temperatura e salinità negli ambienti oceanici.

La propagazione viene modellata attraverso un'espansione concentrica, che considera sia la geometria dello spazio che le caratteristiche fisiche del mezzo. Il fronte d'onda si diffonde radialmente a partire dalla sorgente, mentre l'intensità del segnale è modulata in funzione della distanza, seguendo i principi dell'attenuazione geometrica e tenendo conto delle proprietà dissipative del mezzo, desunte dai valori di intensità contenuti nei dataset.

\begin{figure}[H]
    \centering
    % Primo frame
    \begin{subfigure}{0.3\textwidth}
        \centering
        \includegraphics[width=\linewidth]{images/propagation-1.png}
        \caption{Momento iniziale} % Didascalia specifica
        \label{fig:propagation-frame-1} % Etichetta specifica
    \end{subfigure}
    \hfill % Spazio flessibile
    % Secondo frame
    \begin{subfigure}{0.3\textwidth}
        \centering
        \includegraphics[width=\linewidth]{images/propagation-2.png}
        \caption{Propagazione intermedia} % Didascalia specifica
        \label{fig:propagation-frame-2} % Etichetta specifica
    \end{subfigure}
    \hfill % Spazio flessibile
    % Terzo frame
    \begin{subfigure}{0.3\textwidth}
        \centering
        \includegraphics[width=\linewidth]{images/propagation-3.png}
        \caption{Propagazione avanzata} % Didascalia specifica
        \label{fig:propagation-frame-3} % Etichetta specifica
    \end{subfigure}
    
    \caption{Sequenza di frame illustranti l'evoluzione di un'animazione di propagazione.} % Didascalia generale
    \label{fig:propagation-animation-overview} % Etichetta generale
\end{figure}