\section{Folium}
\label{ch:folium}

\subsection{Facilità d'integrazione e documentazione}  
Folium è una libreria Python che crea mappe interattive basate su Leaflet da notebook e applicazioni web \cite{folium-doc}.  
La documentazione include User Guide con esempi passo‑passo e API Reference \cite{folium-userguide}, mentre il repository GitHub ospita codice, issue e plugin correlati (\textit{xyzservices}, \textit{streamlit-folium}) \cite{folium-github,folium-pypistats}.  
Sono disponibili guide pratiche per integrare Folium in workflow di data science basati su Jupyter Notebook \cite{folium-tutorial,folium-realpython}.  

\subsection{Performance e fluidità dell'interazione}  
Le feature vengono renderizzate come oggetti JavaScript nel DOM di Leaflet, efficienti fino a qualche migliaio di marker.  
Per dataset >10.000 punti si consiglia clustering lato client/server o l'uso di vector tiles \cite{folium-cluster-issue}.  
Plugin come \textit{MarkerCluster} e \textit{FastMarkerCluster} migliorano le prestazioni con dataset voluminosi \cite{folium-cluster-issue,folium-pypistats}.  

\subsection{Supporto a dati vettoriali e raster}  
Folium supporta layer vettoriali (GeoJSON, Choropleth, HeatMap) con popup e tooltip dinamici.  
Raster sono gestiti tramite \texttt{TileLayer}, \texttt{WmsTileLayer} e \texttt{ImageOverlay}, permettendo basemap OSM, mappe satellitari e overlay di immagini georeferenziate.  
Con \textit{Rasterio} è possibile integrare singole bande TIFF in mappe interattive \cite{folium-raster,folium-tutorial,folium-doc}.  

\subsection{Licenza e costi d'uso}  
Folium è rilasciato sotto licenza MIT, utilizzabile, modificabile e ridistribuibile anche commercialmente.  
Dipendenze e plugin sono gestiti via PyPI, con aggiornamenti regolari in linea con le versioni di Leaflet \cite{folium-lic,folium-pypi}.  

\subsection{Compatibilità con altri strumenti e framework}  
Si integra con Jupyter Notebook e JupyterLab generando mappe HTML \say{embeddate}.  
Esistono wrapper per Streamlit (\texttt{streamlit-folium}) e React, spesso tramite \textit{iframe} o REST \cite{folium-github,folium-react}.  
Plugin supportano l'uso con GeoPandas, Pandas e altre librerie di data science \cite{folium-doc,folium-userguide}.  
