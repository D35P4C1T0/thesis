\section{Folium}
\label{ch:folium}

\subsection{Facilità d'integrazione e documentazione}  
Folium è una libreria Python che semplifica la creazione di mappe interattive basate su Leaflet direttamente da notebook e applicazioni web \cite{folium-doc}.
La documentazione ufficiale comprende un User Guide con esempi passo‑passo e un'\textit{API Reference} dettagliata, entrambi disponibili sul sito del progetto \cite{folium-userguide}.  
Il \textit{repository} GitHub ospita codice, issue e un elenco di plugin e pacchetti correlati (\textit{xyzservices}, streamlit‑folium) che estendono le funzionalità di base \cite{folium-github,folium-pypistats}.  
Online è facile reperire delle guide pratiche per integrare Folium in workflow di \textit{data science} basati, per esempio, su Jupyter Notebook \cite{folium-tutorial,folium-realpython}.

\subsection{Performance e fluidità dell'interazione}  
Di default Folium rende le feature come oggetti JavaScript nel DOM di Leaflet: ciò è efficiente per qualche migliaio di marker, ma può diventare lento con dataset superiori a 10.000 punti, come evidenziato da una discussione su StackOverflow riguardo al MarkerCluster, dove, per dataset molto voluminosi, viene consigliato di preprocessare i dati in vector tiles o di applicare tecniche di clustering lato server \cite{folium-cluster-issue}.  

Sono disponibili diversi plugin come \textit{MarkerCluster} e \textit{FastMarkerCluster}, mirati a migliorare l'esperienza utente con il crescere della dimensione del dataset \cite{folium-cluster-issue,folium-pypistats}.  
Il numero di download settimanali (più di 500.000) testimonia la diffusione di Folium e la necessità di \textit{best practice} per gestire prestazioni e scalabilità.

\subsection{Supporto a dati vettoriali e raster}  
Folium supporta nativamente diversi tipi di layer vettoriali (GeoJSON, Choropleth, HeatMap) e mette a disposizione \textit{convenience class} per aggiungere popup e tooltip dinamici.  
Il supporto a raster avviene tramite classi come \texttt{TileLayer}, \texttt{WmsTileLayer} e \texttt{ImageOverlay}, permettendo di utilizzare \textit{basemap} OpenStreetMap, mappe satellitari e overlay di immagini o video georeferenziati.  
Per esigenze avanzate di overlay raster in Jupyter, utilizzando \textit{Rasterio} è possibile integrare singole bande \textit{tiff} in mappe interattive. \cite{folium-raster,folium-tutorial, folium-doc}

\subsection{Licenza e costi d'uso}  
Folium è rilasciato sotto licenza MIT, liberamente utilizzabile, modificabile e ridistribuibile anche in contesti commerciali senza alcun costo di licenza.  
La gestione delle dipendenze e dei plugin avviene via PyPI, con aggiornamenti regolari e un ciclo di rilascio che segue le versioni di Leaflet sottostanti \cite{folium-lic,folium-pypi}.

\subsection{Compatibilità con altri strumenti e framework}  
Folium si integra con Jupyter Notebook e JupyterLab, generando mappe HTML \say{\textit{embeddate}} direttamente nelle celle del notebook.  
Esistono \textit{binding} pacchetti per far girare mappe Folium in applicazioni Streamlit (\texttt{streamlit-folium}) e wrapper per framework frontend come React, sebbene in questi casi si usino spesso \textit{iframe} o comunicazione via REST \cite{folium-github,folium-react}.  
L'ecosistema di plugin ospita esempi per l'uso con GeoPandas, Pandas e altri pacchetti di data science, facilitando l'integrazione in pipeline di elaborazione geospaziale \cite{folium-doc,folium-userguide}.

\newpage