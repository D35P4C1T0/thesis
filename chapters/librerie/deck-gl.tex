\section{Deck.gl}
\label{ch:deckgl}

\subsection{Facilità d'integrazione e documentazione}  
Deck.gl è un framework GPU-powered per la visual exploratory data analysis di grandi dataset geospaziali.  
La documentazione ufficiale (\url{https://deck.gl/docs}) include guide passo‑passo, API reference, esempi e showcase interattivi \cite{deckgl-docs}.  
L'installazione avviene tramite npm/yarn (\texttt{deck.gl}) o CDN \cite{deckgl-npm,deckgl-github}.  

\subsection{Performance e fluidità dell'interazione}  
Grazie a WebGL2 (con supporto sperimentale a WebGPU), Deck.gl mantiene frame rate elevati anche con milioni di punti. Lo \texttt{ScatterplotLayer} rimane fluido fino a ~1M di elementi, degradando controllatamente oltre i 10M \cite{deckgl-performance}.  
Demo mostrano visualizzazioni di 2M di punti e 36k viaggi in tempo reale con interpolazione GPU \cite{deckgl-uber-blog}.  
Il core ottimizza caricamento e aggiornamento dei dati a livello di tile, riducendo pressione sulla CPU e sfruttando efficacemente la GPU \cite{deckgl-github}.  

\subsection{Supporto a dati vettoriali e raster}  
Layer principali: \texttt{GeoJsonLayer}, \texttt{MVTLayer}, \texttt{VectorTileLayer}, che supportano GeoJSON e Mapbox Vector Tiles con clipping e streaming dinamico \cite{deckgl-vector,deckgl-mvtlayer}.  
Per raster tiles, \texttt{RasterTileLayer} e il plugin \texttt{deck.gl-raster} visualizzano immagini satellitari o DEM via WebGL.  
L'architettura a layer composabili facilita estensioni personalizzate e l'uso di moduli WebGL per elaborazioni on-the-fly \cite{deckgl-maptiler,deckgl-raster-plugin}.  

\subsection{Licenza e costi d'uso}  
Deck.gl è rilasciato con licenza MIT, consentendo uso, modifica e ridistribuzione anche commerciale senza royalty, con obbligo di attribuzione \cite{deckgl-license}.  

\subsection{Compatibilità con altri strumenti e framework}  
In React, \texttt{@deck.gl/react} espone \texttt{DeckGL}, hooks per view state ed interazioni \cite{deckgl-react}.  
Altri framework (Vue, Angular) sono supportati tramite binding e plugin non ufficiali \cite{deckgl-community}.
Il sistema a layer lavora con Turf.js e motori 3D come Three.js tramite CustomLayer.  