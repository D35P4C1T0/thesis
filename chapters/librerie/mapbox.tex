\section{Mapbox GL}
\label{ch:mapbox}

\subsection{Facilità d'integrazione e documentazione}  
Mapbox GL JS è una libreria JavaScript client-side che utilizza WebGL per mappe interattive nel browser.  
La documentazione ufficiale comprende guide passo‑passo e API reference aggiornate.  
L'installazione avviene tramite CDN o pacchetto \texttt{mapbox-gl} con npm o yarn \cite{mapbox-docs}.  

\subsection{Performance e fluidità dell'interazione}  
Il rendering GPU-accelerato consente frame rate elevati anche con centinaia di migliaia di feature.  
Prestazioni misurate tramite \emph{render time}, \emph{source update time} e \emph{layer update time} possono essere monitorate tramite logging interno.  
Per dataset molto grandi, l'ottimizzazione dei tileset tramite \emph{tile clipping} e caching client-side migliora l'efficienza \cite{mapbox-vector, article-highperf}.  

\subsection{Supporto a dati vettoriali e raster}  
Mapbox GL JS supporta nativamente \emph{vector tiles} in formato MVT (\texttt{.mvt}) \cite{mapbox-vector}.  
I \emph{raster tiles} da terze parti si aggiungono facilmente definendo una sorgente \texttt{raster} e il layer corrispondente \cite{mapbox-raster}.  

\subsection{Licenza e costi d'uso}  
Dalla versione 2, Mapbox GL JS è distribuito con licenza commerciale, richiedendo un \textit{token} per l'uso in produzione.  
Il modello di pricing si basa sui \emph{map loads} mensili, con soglia gratuita e piani a consumo scalabili \cite{mapbox-license-v2, mapbox-pricing}.  

\subsection{Compatibilità con altri strumenti e framework}  
Per React è disponibile \texttt{react-map-gl}, semplificando l'integrazione \cite{react-map-gl,mapbox-react-tutorial}.  
Per analisi spaziali avanzate si può combinare con Turf.js o estendere mappe 3D tramite Three.js con un CustomLayer.  
SDK nativi per iOS e Android consentono di riutilizzare stili e tileset MVT con API coerenti \cite{mapbox-ios, mapbox-mobile-sdk}.  
