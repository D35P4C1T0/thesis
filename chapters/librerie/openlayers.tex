\section{OpenLayers}
\label{ch:openlayers}

\subsection{Facilità d'integrazione e documentazione}  
OpenLayers offre documentazione dettagliata con guide, esempi interattivi e API reference \cite{openlayers-doc}.  
Codice sorgente e licenza Clear BSD 2‑Clause sono su GitHub; plugin e wrapper non ufficiali sono raccolti in repository come \say{awesome-openlayers} \cite{openlayers-github,awesome-openlayers}.  

\subsection{Performance e fluidità dell'interazione}  
Il supporto WebGL consente rendering GPU per geometrie complesse: \texttt{WebGLVectorLayer} mantiene frame rate elevati con decine di migliaia di punti \cite{openlayers-webgl-example}.  
Tecniche di ottimizzazione includono tiling client-side, gestione dinamica dei dati, clustering e buffer dinamici \cite{openlayers-webgl-workshop,openlayers-issue-perf}.  

\subsection{Supporto a dati vettoriali e raster}  
\textbf{Vector tiles:} supporto nativo a Mapbox Vector Tiles tramite \texttt{VectorTileLayer}, con styling lato client e caricamento tile-by-tile. Esempi OSM MVT mostrano buone performance su dataset medi \cite{openlayers-mapbox-vector-tiles,openlayers-osm-vector-tiles}.  
\textbf{Raster tiles:} supporto a XYZ e WMS tramite \texttt{ol/source/XYZ} e \texttt{ol/source/TileWMS}, semplificando l'uso di basemap come OSM o Geoserver \cite{openlayers-raster-xyz}.  

\subsection{Supporto a WebGL}  
Layer \texttt{WebGLPointsLayer} e \texttt{WebGLVectorLayer} trasferiscono il rendering alla GPU tramite buffer e shader GLSL, consentendo visualizzazioni fluide di grandi dataset \cite{openlayers-webgl-example,openlayers-webgl-workshop}.  

\subsection{Licenza e costi d'uso}  
OpenLayers è rilasciato sotto Clear BSD 2‑Clause, permissiva e senza royalty \cite{openlayers-github,ol-cesium-license}.  
Non ci sono costi di licenza, ma si incoraggia a supportare il progetto tramite donazioni OSGeo \cite{openlayers-doc}.  

\subsection{Compatibilità con altri strumenti e framework}  
Oltre a React, sono disponibili binding per Vue, Angular e GWT.  
L'integrazione con Turf.js è comune per analisi spaziali lato client \cite{awesome-openlayers,openlayers-turf-example}.  
