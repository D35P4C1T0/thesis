\section{OpenLayers}
\label{ch:openlayers}

\subsection{Facilità d'integrazione e documentazione}
OpenLayers mette a disposizione una documentazione ufficiale molto dettagliata, con guide introduttive, esempi interattivi e API reference sul sito del progetto \cite{openlayers-doc}.  
Il codice sorgente e il tracker delle issue sono ospitati su GitHub, dove si trova anche il file di licenza Clear BSD 2‑Clause.
È inoltre attivo un ecosistema di wrapper e plugin non ufficiali raccolti in repository quali \say{awesome-openlayers} \cite{openlayers-github,awesome-openlayers}.

\subsection{Performance e fluidità dell'interazione}
Il supporto WebGL in OpenLayers permette di sfruttare la GPU per il rendering di geometrie complesse: esempi ufficiali mostrano come il layer \texttt{WebGLVectorLayer} mantenga un frame rate elevato anche con decine di migliaia di punti. \cite{openlayers-webgl-example} 
Workshop dedicati illustrano tecniche di ottimizzazione, come il tiling client‑side e la gestione dinamica dei dati, per mantenere interactive performance anche in scenari di grandi dataset.
Discutere problemi di performance su forum come GIS StackExchange rivela strategie quali clustering e buffer dinamici per ridurre il carico di rendering. \cite{openlayers-webgl-workshop,openlayers-issue-perf}

\subsection{Supporto a dati vettoriali e raster}
\subsubsection*{Vector tiles}
OpenLayers supporta nativamente Mapbox Vector Tiles (MVT) tramite il layer \texttt{VectorTileLayer}, consentendo styling lato client e caricamento tile‑by‑tile.
Sono disponibili esempi per OSM Vector Tiles che utilizzano il formato MVT per ottenere ottime prestazioni con dataset di medie dimensioni. \cite{openlayers-mapbox-vector-tiles,openlayers-osm-vector-tiles}

\subsubsection*{Raster tiles}
Il framework gestisce tile raster da sorgenti XYZ e WMS con le classi 

\begin{itemize}
    \item \texttt{ol/source/XYZ} 
    \item \texttt{ol/source/TileWMS}
\end{itemize}

semplificando l'uso di basemap come OpenStreetMap o servizi Geoserver\cite{openlayers-raster-xyz}.  
Per casi avanzati, la documentazione ufficiale risulta un'importante risorsa per ottenere più informazioni sulla riproiezione dei raster.

\subsection{Supporto a WebGL}
Il motore WebGL di OpenLayers, accessibile tramite \texttt{WebGLPointsLayer} e \texttt{WebGLVectorLayer}, sfrutta buffer di vertici e shader GLSL per trasferire gran parte del lavoro di rendering alla GPU, rendendo possibile la visualizzazione di grandi moli di dati con animazioni fluide. \cite{openlayers-webgl-example,openlayers-webgl-workshop}

\subsection{Licenza e costi d'uso}
OpenLayers è rilasciato sotto licenza Clear BSD 2‑Clause, una licenza permissiva che permette uso, modifica e ridistribuzione anche in progetti commerciali senza royalty \cite{openlayers-github,ol-cesium-license}.  
Non sono previsti costi di licenza, ma il progetto invita a donare tramite OSGeo per sostenere il mantenimento del software \cite{openlayers-doc}.

\subsection{Compatibilità con altri strumenti e framework}
Oltre a React, la comunità ha sviluppato binding per Vue, Angular e GWT, elencati nella pagina \say{Useful 3rd party libraries} del sito ufficiale.  
Per analisi spaziale lato client è comune integrare OpenLayers con Turf.js, come mostrato nell'esempio ufficiale \say{\texttt{turf.js}} \cite{awesome-openlayers,openlayers-turf-example}.  

\newpage