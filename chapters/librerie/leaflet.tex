\section{Leaflet}
\label{ch:leaflet}

\subsection{Facilità d'integrazione e documentazione}  
Leaflet offre guida introduttiva, esempi interattivi e un'API reference completa \cite{leaflet-doc}.  
La curva di apprendimento è bassa grazie a metodi semplici come \texttt{L.map()}, \texttt{L.tileLayer()} e \texttt{L.marker()}, e la comunità mantiene numerosi plugin per estendere funzionalità, dal \textit{clustering} dei marker all'integrazione con dataset GeoJSON complessi \cite{react-leaflet}.

\subsection{Performance e fluidità dell'interazione}  
Di default Leaflet usa SVG e elementi DOM, garantendo ottime performance fino a 5.000–10.000 feature\cite{supercluster}.  
Per dataset più voluminosi, plugin come \texttt{Supercluster} raggruppano i punti via \textit{clustering} su R‑tree, riducendo drasticamente i marker renderizzati\cite{supercluster}.  
Estensioni WebGL come \texttt{leafgl} sfruttano la GPU per mantenere interattività fluida anche con decine di migliaia di feature\cite{leafgl}.  

WebGL è un'API JavaScript per il rendering 2D/3D ad alte prestazioni nel browser, sfruttando l'accelerazione hardware della GPU \cite{wiki-webgl}. Il supporto WebGL è fondamentale per gestire grandi dataset geospaziali, trasferendo il carico grafico dalla CPU alla GPU tramite \textit{shader} e buffer dedicati \cite{khronos-webgl}. Senza WebGL, il rendering massivo ricadrebbe sul DOM e sulla CPU, con evidenti rallentamenti.

\subsection{Supporto a dati vettoriali e raster}  
I \textit{raster} rappresentano dati spaziali su griglia di celle, ciascuna con un valore numerico che descrive un fenomeno reale, come temperatura o elevazione, derivati da immagini satellitari o mappe scannerizzate \cite{esri-raster-model,qgis-raster-data}.  
La gestione dei \emph{raster tiles} (es. OpenStreetMap, MapTiler) è immediata con \texttt{L.tileLayer()}\cite{maptiler-raster}.  
Il supporto ai \emph{vector tiles} può essere aggiunto con plugin come \texttt{leaflet-geojson-vt} e \texttt{Leaflet.VectorGrid}, che frammentano file GeoJSON/TopoJSON in tile \textit{client‑side} per un caricamento più efficiente \cite{leaflet-geojson-vt,vectorgrid}.  

\subsection{Licenza e costi d'uso}  
Leaflet è rilasciato sotto licenze BSD2‑Clause e BSD3‑Clause, permissive e compatibili con progetti commerciali senza royalty. Non ci sono canoni d'uso; supporto professionale e SLA a pagamento sono disponibili tramite terze parti \cite{leaflet-doc, leaflet-license}.

\subsection{Compatibilità con altri strumenti e framework}  
Originariamente libreria JavaScript, per React esiste \texttt{ReactLeaflet}, una raccolta di componenti che incapsulano mappe Leaflet in JSX e ne semplificano l'uso \cite{react-leaflet}.
