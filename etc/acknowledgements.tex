\thispagestyle{empty}

\begin{center}
  {\bf \Huge Ringraziamenti}
\end{center}

\vspace{4cm}

\emph{
Ringrazio sinceramente il Professor Paolo Casari per la supervisione e il supporto, fondamentali per l'impostazione e la rifinitura di questo lavoro.
Un sentito ringraziamento anche al Co-Supervisore Mohammad Rasoul Tanhatalab per il prezioso contributo tecnico e la disponibilità dimostrata durante tutto il progetto. Il loro supporto è stato essenziale per la realizzazione della tesi.
}

\;



\emph{
Non potrei mai ringraziare abbastanza la persona con la quale ho condiviso tutto in questi ultimi anni, il mio punto di riferimento e la mia isola felice, Bianca. 
Concludere il percorso universitario è di per sé un traguardo significativo, un misto di sollievo, orgoglio e la consapevolezza di un nuovo inizio. Ma c'è qualcosa che rende questo momento ancora più speciale, quasi magico: condividerlo con la persona amata. Non riesco nemmeno a immaginare questi ultimi anni senza la tua preziosa presenza, che ha reso ogni momento indimenticabile; sei stata la spalla su cui piangere dopo un esame andato male, la motivazione per non mollare quando la stanchezza prendeva il sopravvento, o semplicemente la presenza rassicurante che rendeva ogni sfida un po' meno ardua. 
\noindent\\
Questo traguardo è la celebrazione di un successo congiunto, il coronamento di un periodo intenso che ci ha visti crescere sia individualmente che come coppia. E mentre si chiude un capitolo importante della vostra vita, se ne apre uno nuovo, con progetti e sogni da costruire insieme, forti di un legame che si è dimostrato solido anche sotto la pressione degli studi. Non vedo l'ora di raggiungere tutti i nostri futuri obiettivi nel migliore dei modi, insieme.
}

\;

\emph{Un ringraziamento sentito è rivolto a Mamma e Papà, per l'incondizionato sostegno emotivo ed economico che ha reso possibile il completamento di questo percorso accademico.
Estendo la mia gratitudine a Alessandro, Gina, Mario, Stefania, a tutti gli zii, cugini e familiari, per la costante vicinanza e il prezioso supporto offerto nel corso di questi anni.
\noindent\\
Un grazie speciale va a tutti i miei nonni.
In particolare, i nonni Leda e Franco: durante il mio percorso di studi vi siete spenti con dolcezza. La vostra assenza ha lasciato un segno profondo, ma il ricordo del vostro affetto mi ha accompagnato e sostenuto in ogni passo.
Mi mancate, ma il vostro amore resta la mia forza più grande.
Questa tesi è dedicata anche a voi. Spero di avervi reso orgogliosi.}

\;

\emph{Non posso che ringraziare tutti i miei compagni di viaggio universitari, tra cui Marco, Federico, Edoardo, Riccardo. La vostra presenza ha alleggerito i momenti di studio più intensi e ha reso ogni sfida più affrontabile. Grazie per aver condiviso non solo il percorso accademico, ma anche i preziosi momenti di spensieratezza che hanno arricchito profondamente la mia esperienza.
\noindent\\
Un grazie di cuore va a tutti i miei amici di Brescia. La vostra presenza, costante o saltuaria, è stata un punto di riferimento importante, capace di alleggerire i momenti difficili e rendere più piacevoli quelli sereni. Vi ringrazio per aver condiviso con me un pezzo di questo percorso.
}

\vspace{2em}

\emph{Infine, un piccolo grazie, quasi sottovoce, va a chi ha scritto queste pagine. Per la tanta fatica e per non aver mai mollato.}